
\documentclass{beamer}
\usepackage[spanish]{babel}
\usepackage{amsthm}
\usepackage{bbding}
\usepackage[utf8]{inputenc}
\usepackage{verbatim}
%\usepackage{enumitem}

\mode<presentation> {
\usetheme{Madrid}
}
\usepackage{color}
\usepackage{graphicx} % Allows including images
\usepackage{booktabs} % Allows the use of \toprule, \midrule and \bottomrule in tables
\setbeamertemplate{blocks}[rounded][shadow=true]
%----------------------------------------------------------------------------------------
%	TITLE PAGE
%----------------------------------------------------------------------------------------

\title[Presentación del TP2]{Web Semántica} % The short title appears at the bottom of every slide, the full title is only on the title page

\author{Bases de Datos} % Your name
\institute[] % Your institution as it will appear on the bottom of every slide, may be shorthand to save space
{
Federico Allocati, Sabrina Izcovich, Santiago Pernigotti, Germán Romano\\ % Your institution for the title page
\medskip
%\textit{john@smith.com} % Your email address
}
\date{\today} % Date, can be changed to a custom date

\begin{document}

\begin{frame}
\titlepage % Print the title page as the first slide
\end{frame}

\begin{frame}
\frametitle{Contenidos} % Table of contents slide, comment this block out to remove it
\tableofcontents % Throughout your presentation, if you choose to use \section{} and \subsection{} commands, these will automatically be printed on this slide as an overview of your presentation
\end{frame}

%----------------------------------------------------------------------------------------
%	PRESENTATION SLIDES
%----------------------------------------------------------------------------------------

%------------------------------------------------
\section{Introducción} 
%------------------------------------------------
\begin{frame}
\frametitle{Material Analizado}
\begin{itemize}
\item \textbf{Optimizing Reformulated RDF Queries}, Damián A. Bursztyn, 2013.

\item \textbf{Introduction to the Special Issue on Semantic Web Data Management}, Roberto De Virgilio, 2011.
\end{itemize}
\end{frame}


\begin{frame}
\frametitle{Presentación de conceptos}
\begin{itemize}
\item La \textbf{Web Semántica} es una extensión de la Web tradicional, con especificaciones manejadas por el \textbf{W3C} (\textit{World Wide Web Consortium}).

\item Los estándares más utilizados son \textbf{RDF} (\textit{Framework de Descripción de Recursos}) y \textbf{SPARQL} (\textit{Lenguaje de Queries RDF}).

\item \textbf{RDF} se usa para el almacenamiento, representación, e intercambio de datos y \textbf{SPARQL} es el lenguaje estandarizado de consulta sobre grafos \textbf{RDF}.
\end{itemize}
\end{frame}

%------------------------------------------------
\section{Web Semántica} 
%------------------------------------------------
\begin{frame}
\frametitle{¿Qué es la Web Semántica?} 
\begin{block}{Web Semántica}
\begin{itemize}
\item<1-> La Web tradicional contiene una enorme cantidad de información, pero al no estar estructurado, se torna dificil de explotar.
\item<2-> Para solucionar esto, se propone la Web Semantica, con el foco en el significado del contenido, en lugar de en la visualización, estructurandolo para que sea facil de procesar mediante software.
\item<3-> El software es capaz de procesar el contenido, ``razonar'' con este, combinarlo y realizar deducciones lógicas sobre su semántica.
\item<4-> Utiliza una infraestructura común mediante la cual es posible compartir, procesar y transferir información de forma sencilla.
\item<5-> Otorga una solución a problemas habituales en la búsqueda de información ocasionados por la sobrecarga de información y heterogeneidad de fuentes de información.
\end{itemize}
\end{block}
\end{frame}
%------------------------------------------------
\begin{frame}
\frametitle{Ejemplo} 
\begin{figure}[H] %[h] Aqui [b] para button [t] para top
\begin{center}
\includegraphics[width=340pt]{./imgs/resultadoNormal.png}
\caption{Resultados obtenidos con un buscador normal.}
\end{center}
\end{figure}
\end{frame}
%------------------------------------------------
\begin{frame}
\frametitle{Ejemplo} 
\begin{figure}[H] %[h] Aqui [b] para button [t] para top
\begin{center}
\includegraphics[width=340pt]{./imgs/resultadoSemantico.png}
\caption{Resultados obtenidos con un buscador semántico.}
\end{center}
\end{figure}
\end{frame}
%------------------------------------------------
\section{RDF}
\begin{frame}[fragile] % Need to use the fragile option when verbatim is used in the slide
\frametitle{¿Qué es RDF?}
\begin{block}{Resource Description Framework}
\begin{itemize}
\item<1-> Es un grafo dirigido etiquetado para representar información en la Web. Los nodos son lo que se relaciona en la Web.
\item<2-> En particular, es el formato para modelar el intercambio de datos en la Web Semántica. 
\item<3-> Facilita la fusión de datos con esquemas de organización diferentes.
\item<4-> Permite unir estructuras en la Web dando un nombre a las relaciones entre nodos.
\item<5-> Su utilización permite estructurar los datos a ser fusionados, expuestos y compartidos a través de diferentes aplicaciones.
\end{itemize}
\end{block}
\end{frame}

%------------------------------------------------
\begin{frame}[fragile]
\frametitle{Ejemplo}
\textit{``Javier Perez es el autor del documento cuya URL es http://www.dc.uba.ar/examen.pdf. Su mail es jperez@dc.uba.ar y su teléfono +54(11)4563-4567.''}
\begin{verbatim}
<?namespace href="http://dc.uba.ar/info-bibliografia" as="bib"?> 
<RDF:serialization> 
  <RDF:assertions href="http://dc.uba.ar/examen.pdf"> 
    <bib:autor> 
      <RDF:resource> 
        <bib:nombre>Javier Perez</bib:nombre> 
        <bib:mail>jperez@dc.uba.ar</bib:mail> 
        <bib:telefono>+54(11)4563-4567</bib:telefono> 
      </RDF:resource> 
    </bib:autor> 
  </RDF:assertions> 
</RDF:serialization>

\end{verbatim}
\end{frame}

%------------------------------------------------
\begin{frame}[fragile]
\frametitle{Ejemplo} 
\begin{center}
\begin{figure}[H] %[h] Aqui [b] para button [t] para top
\includegraphics[width=300pt]{./imgs/rdfModel.png}
\caption{Modelo RDF.}
\end{figure}
Los nodos son las elipses, los ejes la relación y los rectángulos son strings.
\end{center}
\end{frame}
%------------------------------------------------
\section{BGP}
\begin{frame}
\frametitle{Queries BGP}
\begin{exampleblock}{Basic Graph Pattern}
\begin{itemize}
\item<1-> Las consultas BGP son un subconjunto del lenguaje de consultas SPARQL W3C para RDF. 
\item<2-> El mismo se corresponde con la clase de familias de consultas conjuntivas de las bases de datos relacionales. 
\item<3-> Conforman las consultas SPARQL más utilizadas a nivel mundial.
\item<4-> Están compuestas por un conjunto de variables distinguidas, la cabeza y el conjunto de patrones de tripla (los BGP).
\end{itemize}
\end{exampleblock}
\end{frame}

\begin{frame}
\frametitle{Relación BGP-SPARQL}
\begin{figure}[H] %[h] Aqui [b] para button [t] para top
\begin{center}
\includegraphics[width=340pt]{./imgs/conceptos.png}
\caption{Relación entre conceptos.}
\end{center}
\end{figure}
\end{frame}
%------------------------------------------------
\begin{frame}
\frametitle{Ejemplo de BGP}
\begin{figure}
        \setbox0\hbox{%
                \includegraphics[width=.35\textwidth]{./imgs/ejemploBGP-01.png}%
        }%
        \setbox2\hbox{%
                \includegraphics[width=.35\textwidth]{./imgs/ejemploBGP-01_grafico.png}%
        }%
        \ifdim\ht0>\ht2
                \setbox0\hbox{%
                        \includegraphics[height=\ht2]{./imgs/ejemploBGP-01.png}%
                }%
        \else
                \setbox2\hbox{%
                        \includegraphics[height=\ht2]{./imgs/ejemploBGP-01_grafico.png}%
                }%
        \fi
        \noindent
        \parbox{.35\textwidth}{%
                \centering
                \unhbox0
                \caption{Consulta RDF}
                \label{fg:methods}
        }%
        \hfil
        \parbox{.35\textwidth}{%
                \centering
                \unhbox2
                \caption{Grafo RDF}
                \label{fg:method_detail}
        }
\end{figure}
\vspace*{1.2cm}
Consulta BGP que devuelve true si existe un director que protagonizó su película.
\end{frame}

\begin{frame}
\frametitle{Ejemplo de BGP}
\begin{figure}
        \setbox0\hbox{%
                \includegraphics[width=.45\textwidth]{./imgs/ejemploBGP-02.png}%
        }%
        \setbox2\hbox{%
                \includegraphics[width=.45\textwidth]{./imgs/ejemploBGP-02_grafico.png}%
        }%
        \ifdim\ht0>\ht2
                \setbox0\hbox{%
                        \includegraphics[height=\ht2]{./imgs/ejemploBGP-02.png}%
                }%
        \else
                \setbox2\hbox{%
                        \includegraphics[height=\ht2]{./imgs/ejemploBGP-02_grafico.png}%
                }%
        \fi
        \noindent
        \parbox{.45\textwidth}{%
                \centering
                \unhbox0
                \caption{Consulta RDF}
                \label{fg:methods}
        }%
        \hfil
        \parbox{.45\textwidth}{%
                \centering
                \unhbox2
                \caption{Grafo RDF}
                \label{fg:method_detail}
        }%
\end{figure}
\vspace*{1.2cm}
Consulta que devuelve los actores que trabajaron en una película dirigida por un francés.
\end{frame}

%------------------------------------------------
\section{RDF y RDBMs}
\begin{frame}
\frametitle{RDF y Sistemas de Gestión de BD Relacionales}

\begin{block}{RDF y RDBMs}
\begin{itemize}
\item Es posible transformar una consulta BGP en otra capaz de ser evaluada por RDBMS.
\vspace{1 mm}
\item Se almacenan las triplas dentro de tablas y se reformula la consulta BGP por una equivalente.
\vspace{1 mm}
\item $q(x) : s_{1} p_{1} o_{1}, ..., s_{n} p_{n} o_{n}$ es equivalente a $q(x): \wedge ^n _{i = 1} Triples(s_{i}, p_{i}, o_{i})$
\vspace{1 mm}
\item Para resolver la consulta se emplean dos técnicas de manejo de vinculos RDF: \textbf{Saturación o Reformulación}.
\end{itemize}
\end{block}
\end{frame}
%------------------------------------------------

\begin{frame}
\frametitle{Ejemplo - Saturación}

\textbf{Nuestro Dataset}:
\begin{itemize}
\item Hechos = $\lbrace$ \textit{$"$Sócrates es humano$"$} $\rbrace$
\item Restricciones = $\lbrace$ \textit{$"$Todo humano es mortal$"$} $\rbrace$
\end{itemize}

\vspace{1 mm}

\textbf{La query}: $q$ = \textit{$"$Devolver todos los mortales$"$}.

\vspace{8 mm}

\begin{exampleblock}{Saturación}

\begin{itemize}
\item Hace explícitos los datos implícitos, a partir de las restricciones.
\item Se aplica una única vez a toda la BD (la extiende).
\end{itemize}
\hspace{1cm}$\Rightarrow$ \textbf{Se agrega el hecho \textit{$"$Sócrates es mortal$"$} a la BD}.

\end{exampleblock}

\end{frame}

%------------------------------------------------

\begin{frame}
\frametitle{Ejemplo - Reformulación}

\textbf{Nuestro Dataset}:
\begin{itemize}
\item Hechos = $\lbrace$ \textit{$"$Sócrates es humano$"$} $\rbrace$
\item Restricciones = $\lbrace$ \textit{$"$Todo humano es mortal$"$} $\rbrace$
\end{itemize}

\vspace{1 mm}

\textbf{La query}: $q$ = \textit{$"$Devolver todos los mortales$"$}.

\vspace{7 mm}

\begin{exampleblock}{Reformulación}

\begin{itemize}
\item Se reformula la query teniendo en cuenta las restricciones.
\item Se aplica a cada query, la BD no se modifica.
\end{itemize}
\hspace{1cm} $\Rightarrow$ \textbf{$q$ = $"$Todos los mortales$"$} \\
\hspace{1.8cm} \textbf{$\leadsto$ $q'$ = $"$Todos los mortales y todos los humanos$"$}.

\end{exampleblock}

\end{frame}


%------------------------------------------------

\begin{frame}
\frametitle{RDF y RDBMs}

\begin{figure}[h]
\begin{center}
\includegraphics[width=200pt]{./imgs/saturacion}
\caption{Visión general del proceso de respuesta de una consulta basada en saturación.}
\end{center}
\end{figure}

\begin{figure}[H]
\begin{center}
\includegraphics[width=200pt]{./imgs/reformulacion}
\caption{Visión general del proceso de respuesta de una consulta basada en reformulación.}
\end{center}
\end{figure}
\end{frame}

%------------------------------------------------
\section{Reformulación de Queries}
\begin{frame}
\frametitle{Reformulación de Queries}
\begin{itemize}
\item[]<1->\begin{block}{Problema}
Las bases de datos para Web Semántica presentan datos implícitos que las RDBMs no tienen en cuenta a la hora de resolver las consultas.
\end{block}

\item[]<2->\begin{exampleblock}{Una Solución}
Se aplican técnicas de \underline{Reformulación} a la consulta entrante para luego traducirla a una consulta SQL que, al ser evaluada por el RDBMs, devuelve la respuesta completa.
\end{exampleblock}

\item[]<3->\begin{center}
En la tesis de Damián Bursztyn se proponen algunas heurísticas que mejoran la performance del algoritmo de reformulación, que tiene complejidad exponencial.
\end{center}
\end{itemize}
\end{frame}	

%------------------------------------------------


\begin{frame}
\Huge{\centerline{¿Preguntas?}}
\end{frame}

%----------------------------------------------------------------------------------------

\end{document} 
