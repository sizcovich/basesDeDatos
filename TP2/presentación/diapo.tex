%%%%%%%%%%%%%%%%%%%%%%%%%%%%%%%%%%%%%%%%%
% Beamer Presentation
% LaTeX Template
% Version 1.0 (10/11/12)
%
% This template has been downloaded from:
% http://www.LaTeXTemplates.com
%
% License:
% CC BY-NC-SA 3.0 (http://creativecommons.org/licenses/by-nc-sa/3.0/)
%
%%%%%%%%%%%%%%%%%%%%%%%%%%%%%%%%%%%%%%%%%

%----------------------------------------------------------------------------------------
%	PACKAGES AND THEMES
%----------------------------------------------------------------------------------------

\documentclass{beamer}
\usepackage[spanish]{babel}
\usepackage{amsthm}
\usepackage{bbding}
\usepackage[utf8]{inputenc}
%\usepackage{enumitem}

\mode<presentation> {

% The Beamer class comes with a number of default slide themes
% which change the colors and layouts of slides. Below this is a list
% of all the themes, uncomment each in turn to see what they look like.

%\usetheme{default}
%\usetheme{AnnArbor}
%\usetheme{Antibes}
%\usetheme{Bergen}
%\usetheme{Berkeley}
%\usetheme{Berlin}
%\usetheme{Boadilla}
%\usetheme{CambridgeUS}
%\usetheme{Copenhagen}
%\usetheme{Darmstadt}
%\usetheme{Dresden}
%\usetheme{Frankfurt}
%\usetheme{Goettingen}
%\usetheme{Hannover}
%\usetheme{Ilmenau}
%\usetheme{JuanLesPins}
%\usetheme{Luebeck}
\usetheme{Madrid}
%\usetheme{Malmoe}
%\usetheme{Marburg}
%\usetheme{Montpellier}
%\usetheme{PaloAlto}
%\usetheme{Pittsburgh}
%\usetheme{Rochester}
%\usetheme{Singapore}
%\usetheme{Szeged}
%\usetheme{Warsaw}

% As well as themes, the Beamer class has a number of color themes
% for any slide theme. Uncomment each of these in turn to see how it
% changes the colors of your current slide theme.

%\usecolortheme{albatross}
%\usecolortheme{beaver}
%\usecolortheme{beetle}
%\usecolortheme{crane}
%\usecolortheme{dolphin}
%\usecolortheme{dove}
%\usecolortheme{fly}
%\usecolortheme{lily}
%\usecolortheme{orchid}
%\usecolortheme{rose}
%\usecolortheme{seagull}
%\usecolortheme{seahorse}
%\usecolortheme{whale}
%\usecolortheme{wolverine}

%\setbeamertemplate{footline} % To remove the footer line in all slides uncomment this line
%\setbeamertemplate{footline}[page number] % To replace the footer line in all slides with a simple slide count uncomment this line

%\setbeamertemplate{navigation symbols}{} % To remove the navigation symbols from the bottom of all slides uncomment this line
}
\usepackage{color}
\usepackage{graphicx} % Allows including images
\usepackage{booktabs} % Allows the use of \toprule, \midrule and \bottomrule in tables
\setbeamertemplate{blocks}[rounded][shadow=true]
%----------------------------------------------------------------------------------------
%	TITLE PAGE
%----------------------------------------------------------------------------------------

\title[Presentación del TP2]{Optimización de consultas RDF reformuladas} % The short title appears at the bottom of every slide, the full title is only on the title page

\author{Federico Allocati, Sabrina Izcovich, Santiago Pernigotti, Germán Romano} % Your name
\institute[] % Your institution as it will appear on the bottom of every slide, may be shorthand to save space
{
Departamento de Computación\\ % Your institution for the title page
\medskip
%\textit{john@smith.com} % Your email address
}
\date{Viernes 20 de junio de 2015} % Date, can be changed to a custom date

\begin{document}

\begin{frame}
\titlepage % Print the title page as the first slide
\end{frame}

\begin{frame}
\frametitle{Contenidos} % Table of contents slide, comment this block out to remove it
\tableofcontents % Throughout your presentation, if you choose to use \section{} and \subsection{} commands, these will automatically be printed on this slide as an overview of your presentation
\end{frame}

%----------------------------------------------------------------------------------------
%	PRESENTATION SLIDES
%----------------------------------------------------------------------------------------

%------------------------------------------------
\section{Introducción} % Sections can be created in order to organize your presentation into discrete blocks, all sections and subsections are automatically printed in the table of contents as an overview of the talk
%------------------------------------------------
\begin{frame}
%\frametitle{¿Dónde Estamos?}
%\begin{columns}[c] % The "c" option specifies centered vertical alignment while the "t" option is used for top vertical alignment

%\column{.45\textwidth} % Left column and width
%\begin{itemize}
%\item Ya repasamos elementos de álgebra lineal.
%\item Refrescamos las nociones de matriz, inversibilidad y determinante.
%\item ¡Ahora veamos un ejercicio!
%\end{itemize}

%\column{.5\textwidth} % Right column and width
%\includegraphics[width=200pt]{checklist.jpg}
%\end{columns}
\end{frame}


%------------------------------------------------
\section{Repaso} % A subsection can be created just before a set of slides with a common theme to further break down your presentation into chunks

%------------------------------------------------

\begin{frame}
%\frametitle{Pero antes}
%\begin{block}{Matriz inversible}
%Sea $A \in \mathbb{R}^{nxn}$, se dice que A es inversible si existe una matriz $A^{-1} \in \mathbb{R}^{nxn}$ tal que
%$$AA^{-1} = I\ \land\ A^{-1}A = I$$
%\end{block}

\end{frame}

%------------------------------------------------


%------------------------------------------------
%\section{¡Ejercicio!}
%------------------------------------------------

\begin{frame}[fragile] % Need to use the fragile option when verbatim is used in the slide
%\frametitle{Y ahora...}
%\begin{block}{Ejercicio}
%Una matriz $A \in \mathbb{R}^{nxn}$ se dice nilpotente si $A^{k}$ = 0 para algún $k \in \mathbb{N}$. Probar que si $A$ es nilpotente $\implies$ $A$ no es inversible.
	
%\begin{enumerate}
%\item $A$ no es inversible.
%\item $I - A$ es inversible.
%\end{enumerate}
%\end{block}
\end{frame}

%------------------------------------------------
% \begin{frame}
% \Huge{\centerline{¿Y eso cómo se prueba?}}
% \begin{figure}[H] %[h] Aqui [b] para button [t] para top
% \begin{center}
% \includegraphics[width=150pt]{surprise.png}
% \end{center}
% \end{figure}
% \end{frame}



\begin{frame}
% \frametitle{Ejercicio}
% \begin{exampleblock}{Resolución}
% Queremos ver que si $A^{h}$ = 0 para algún $h \in \mathbb{N}$ entonces $A$ no es inversible.\newline

% Supongamos que lo es,

% \begin{itemize}
% \item<2-> Sea $k$ un índice de nilpotencia de A.
% \item<3-> Multipliquemos a $A^k$ por la inversa de $A$ $k$ veces.
% \item<4-> $A^k(A^{-1})^k$ = $A...AA^{-1}...A^{-1}$
% \begin{description}%[font=\normalfont]
% \item<5->[\textcolor{black}{$\Rightarrow$}] 0 = $I$. \textbf{Abs!}
% \end{description}
% \item<6-> Entonces $A$ no es inversible. \Checkmark
% \end{itemize}

% \end{exampleblock}
\end{frame}

%\begin{frame}
%\frametitle{Ejercicio}
%\begin{exampleblock}{Resolución 2}
%Queremos ver que si $A^{k}$ = 0 para algún $k\ \in\ \mathbb{N}$ entonces $I - A$ es inversible.\newline

%Probemos construyendo la inversa, pero antes...

%\end{exampleblock}
%\end{frame}

%\begin{frame}
%\frametitle{Ejercicio 8}
%Veamos una propiedad de la práctica:
%\begin{block}{Propiedad}
%Sea $A \in \mathbb{R}^{nxn}$ y $m \in \mathbb{N}$, entonces vale que:
%$$(I-A)(I+A+...+A^m) = I - A^{m+1}$$
%\end{block}
%\end{frame}

%\begin{frame}
%\frametitle{Ejercicio}
%\begin{exampleblock}{Resolución 2}
%Queremos ver que si $A^{k}$ = 0 para algún $k\ \in\ \mathbb{N}$ entonces $I - A$ es inversible.\newline

%\begin{itemize}
%\item<1-> Probemos construyendo la inversa:

%\item<2-> Si usamos la propiedad,
%\begin{description}%[font=\normalfont]
%\item<3-> $(I - A)(I + A +...+ A^{k-1})$
%\item<4-> $= I + A +...+ A^{k-1} - A -...- A^{k-1} - A^k$
%\item<5-> $= I - A^k$
%\item<6-> $= I$
%\end{description}

%\item<3-> $(I - A)(I + A +...+ A^{k-1}) = I + A +...+ A^{k-1} - A -...- A^{k-1} - A^k$
%\item<4-> $= I - A^k= I$
	
%\item<7-> Como existe la inversa entonces $I - A$ es inversible. \Checkmark
%\end{itemize}

%\end{exampleblock}
%\end{frame}

%------------------------------------------------


%------------------------------------------------


%------------------------------------------------

%------------------------------------------------

%------------------------------------------------

\begin{frame}
\Huge{\centerline{¿Preguntas?}}
\end{frame}

%----------------------------------------------------------------------------------------

\end{document} 
