\documentclass[10pt,a4paper]{article}
\usepackage[utf8]{inputenc}
\usepackage[spanish]{babel}
\usepackage{a4wide}
\usepackage[sinEntregas]{caratula}
\usepackage{ulem}
\usepackage{marginnote}
\usepackage{fancyhdr}
\usepackage{lastpage}
\usepackage{float}
\usepackage{tikz}
\usepackage{mattens}

\pagestyle{fancy}
\thispagestyle{fancy}
\addtolength{\headheight}{1pt}
\lhead{Bases de Datos}
\rhead{TP2}
\cfoot{\thepage /\pageref{LastPage}}
\renewcommand{\thesubsubsection}{\thesubsection.\alph{subsubsection}}

\title{Bases de Datos - TP 2}
\author{Bases de Datos, DC, UBA.}

\begin{document}

\fecha{20 de junio de 2015}

\materia{Bases de Datos}
%\submateria{Trabajo Pr\'actico Nº1}
\titulo{Web Semántica}

\integrante{Allocati, Federico}{682/11}{fede.allocati@gmail.com}
\integrante{Izcovich, Sabrina}{550/11}{sizcovich@gmail.com}
\integrante{Pernigotti, Santiago}{870/11}{spernigotti@hotmail.com}
\integrante{Romano, Germán}{786/11}{romano.german@live.com.ar}

\maketitle

\tableofcontents

\newpage

\section{Introducción}

El trabajo práctico se basa en un análisis enfocado en el tema de \textit{Semántica Web}. Para la realización del mismo, basamos nuestra investigación en la tesis\footnote{http://dc.uba.ar/inv/tesis/licenciatura/2013/bursztyn.pdf} presentada por Damián A. Bursztyn en el año 2013 sobre Optimización de consultas RDF reformuladas.

En lo que sigue, explicaremos los conceptos de Semántica Web y sus problemáticas, como también de RDF, mecanismo que ayuda a convertir la Web en una infraestructura global en la que es posible compartir, y reutilizar datos y documentos entre diferentes tipos de usuarios.

\newpage
\section{Semántica Web}
La semántica web consiste en una Semántica Extendida dotada de mayor significado gracias a una información mejor definida. Su utilidad principal es encontrar soluciones a problemas habituales en la búsqueda de información gracias a la utilización de una infraestructura común mediante la cual se procesa y transfiere la información de una manera sencilla. Esta Web extendida y basada en el significado, se apoya en lenguajes universales que resuelven los problemas ocasionados por una Web carente de semántica en la que, en ocasiones, el acceso a la información se convierte en una tarea difícil y frustrante.

\section{Sémantica Web y Manejo de Datos}
El exponencial crecimiento de los datos en la web deja en claro la necesidad de complementar el estudio de la web Semántica con dos disciplinas bien desarrolladas, modelado conceptual y manejo de datos. 
\\
Modelado conceptual se encarga del desarrollo de técnicas avanzadas y herramientas que permiten la representación precisa y razonamiento de artefactos que modelan objetos e información del mundo real. Manejo de datos, por otro lado, se encarga del desarrollo de técnicas para el eficiente almacenamiento, consultas, recuperación y manejo de grandes volúmenes de información de diferente naturaleza, es decir, datos relacionales, documentos XML, blogs, etc.
\\
A continuación se presentan contribuciones que abarcan las tres áreas, semántica web, modelado conceptual y manejo de datos.
\\\\
\textbf{Quality-aware} propone algunas estrategias para diseñar funciones de similitud que capturan el grado de libertad para la equivalencia de dos entidades dadas. El método se basa en la combinación de múltiples evidencias, con la ayuda de calidad estimada de valores individuales similares y con la particular atención a la perdida de información, que es común en el contexto de la web. 
\\\\
\textbf{Structured Data Clouding across Multiple Webs} introduce la noción de inCloud, una colección de recursos web relacionados construidos para una entidad objetivo de interés mediante la distinción, también en una forma visual, de que tan prominente es cada recurso web recibido con respecto a la entidad dada
\\\\
\textbf{Ontological Query Answering under Expressive Entity-Relationship Shemata} contribuye a reducir la brecha entre modelos conceptuales y razonamiento semántico, abordando el problema de respuestas a consultas conjuntivas bajo restricciones representadas por esquemas expresados en ER+, una extensión del modelo entidad-relación.
\\\\
\textbf{PEST: Fast Approximate Keyword Search in Semantic Data using Eigenvector-based Term Propagation} presenta un novedoso enfoque a las consultas aproximadas de datos sobre grafos-estructurados RDF que propagan términos ponderados entre ítems de datos que son conectados en la estructura de datos.
\\\\
\textbf{An Ontology-Based Retrieval System Using Semantic Indexing} introduce un sistema para ontologías basadas en extracción de información.
\\\\
\textbf{A step forward is taken in PoweR-Gen: A Power-law Based Generator of RDFS Schemas} presenta el primer generador de esquemas RDF, teniendo en cuenta las características exhibidas por los esquemas reales SW.


\newpage
\section{Problemas de la Web Semántica}
En el último tiempo, el uso de Web Semántica creció enormemente e impulsó la necesidad de emplear técnicas eficientes y escalables para responder a las consultas RDF sobre una gran cantidad de datos heterogéneos. Una posible solución a este problema consiste en traducir las consultas RDF en consultas SQL para ejecutarlas en los sistemas de gestión de bases de datos relacionales (RDBMS). Sin embargo, las bases de datos para Web Semántica complican a las tecnologías clásicas de gestión de datos que no tienen en cuenta los datos implícitos durante la evaluación de consultas. Una solución a esto es reformular la consulta entrante para luego traducirla en una consulta SQL que, al ser evaluada por el RDBMS, devuelve las respuestas completas. El problema que esta solución presenta es de rendimiento debido a la longitud sintáctica de las consultas SQL que resultan de reformulación. Los RDBMSs no son capaces de optimizar eficientemente estas consultas, por lo que en algunos casos fallan o registran tiempos elevados de evaluación.

\newpage
\section{RDF}
Un set de datos RDF consiste en datos explícitos e implícitos presentes en la base de datos, dados por restricciones semánticas que deben cumplir. Dichos datos son obtenidos por un proceso que considera las restricciones para inferir todas las posibles consecuencias de la base de datos existente.

\newpage

\section{RDF y RDBMS}
Una alternativa frente a las tradicionales tablas de datos relacionales que incluyen valores NULL en ellas, son las tablas-V. Estas permiten el uso de variables en sus filas, la cual es una característica importante porque posibilita el uso de juntas para valores desconocidos, dado que es posible utilizar la misma variable en diferentes filas.
\\\\
Utilizando la evaluación del sistema de gestión de base de datos relacionales (RDBMS), podemos obtener el conjunto de respuestas de consultas BGP de la siguiente forma:

Dado un conjunto de datos $D$, podemos almacenarlo todo en una tabla-V de la forma Triples(s,p,o), guardando los triples de $D$ dentro de la tabla como filas, y además convirtiendo los nodos blancos en variables. Entonces, dado una consulta BGP de la forma $q(x) : s_{1} p_{1} o_{1}, ..., s_{n} p_{n} o_{n}$, se puede reescribir como una consulta conjuntiva para la evaluación del RDBMS del siguiente modo:
\\
$q(x): \wedge ^n _{i = 1} Triples(s_{i}, p_{i}, o_{i})$
\\\\
Mediante dos técnicas establecidas para el manejo de vinculación RDF, llamadas saturación y reformulación, podemos calcular la respuesta de la consulta $q$ en $D$.
\\\\
\textbf{Saturación}: Se evalúa la consulta $q(x): \wedge ^n _{i = 1} Triples(s_{i}, p_{i}, o_{i})$ en la tabla Triples que contenga la clausura de $D$ (todos los datos implícitos se encuentran dentro de la tabla). Esta técnica calcula la clausura del conjunto de datos usando las reglas de vinculación. Sus desventajas son que para calcular el conjunto de datos completo requiere tiempo de cómputo y espacio de almacenamiento, y debe ser recalculado por cada actualización realizada.
\\
\begin{figure}[h]
\begin{center}
\includegraphics[width=270pt]{imgs/saturacion}
\caption{Visión general del proceso de respuesta de una consulta basada en saturación.}
\end{center}
\end{figure}
\\
\textbf{Reformulacion}: Se evalúa $q'(x): \wedge ^n _{i = 1} Triples(s_{i}, p_{i}, o_{i})$ en la tabla Triples que contenga a $D$, donde $q'$ es la reformulación de la consulta $q$. La reformulación se realiza utilizando reglas de vinculación inmediatas, de manera tal que la consulta $q'$ sobre $D$ sea equivalente a $q$ sobre la clausura de $D$. La principal ventaja de este método es que no se ve afectado por las actualizaciones, ya que no es necesario recalcular la clausura del conjunto de datos. 

\begin{figure}[H]
\begin{center}
\includegraphics[width=270pt]{imgs/reformulacion}
\caption{Visión general del proceso de respuesta de una consulta basada en reformulación.}
\end{center}
\end{figure}


\section{Reformulación de Consultas}

Dada una query q y un \textit{dataset} D, queremos reformular q con respecto al esquema D en otra query q$'$ tal que la evaluación de q$'$ contra los datos de D (es decir, q$'$(D$_{data}$)) sea el conjunto respuesta completo de q contra D (o sea, q(D$^{\infty}$)).

\subsection{Consultas Parcialmente Instanciadas}

Sea $q(\bS{x}) :- t_{1},...,t_{n}$ una query y $\sigma$ un mapeo de un subconjunto de variables y \textit{blank nodes} de q a algunos valores (constantes, URIs o \textit{blank nodes}).\\

Una \textbf{consulta parcialmente instanciada} con respecto a q, notada $q_{\sigma}$, es una query $q_{\sigma}(\bS{x}_{\sigma}) :- (t_{1},...,t_{n})_{\sigma}$ donde el mapeo $\sigma$ fue aplicado tanto en las variables de la cabeza como en las variables del cuerpo de q. En el caso $\sigma = \emptyset$, vale $q_{\sigma} = q$.\\

Dada una consulta parcialmente instanciada $q_{\sigma}(\bS{x}_{\sigma}) :- (t_{1},...,t_{n})_{\sigma}$ cuyo conjunto de variables y \textit{blank nodes} es VarBl($q_{\sigma}$) y un \textit{dataset} D cuyo conjunto de valores es Val(D), la evaluación de $q_{\sigma}$ contra D es:\\
\includegraphics[scale=0.45]{imgs/01.png}\\

El conjunto respuesta completo de $q_{\sigma}$ contra D es $q_{\sigma}(D^{\infty})$.

\subsection{Reglas de Reformulación}

A continuación se muestran las 13 reglas de reformulación, cada una definiendo una transformación de la forma $\frac{input}{output}$, para una consulta parcialmente instanciada $q_{\sigma}$ con respecto a un \textit{dataset} D:

\begin{center}
\includegraphics[scale=0.8]{imgs/02.png}
\end{center}

Se pueden distinguir dos grupos de reglas:
\begin{itemize}

\item Las reglas 2.5 hasta 2.13 reformulan la query bindeando una de las variables de la query a la propiedad RDF $\tau$ ó a una clase o propiedad definida en el \textit{dataset}.

\item Las reglas de 2.14 a 2.17 alteran las queries reemplazando una de sus triplas por otra, mediante el uso de esquemas.

\end{itemize}

\subsection{Resolución de Consultas Basada en Reformulación}

La resolución de consultas basada en reformulación reformula una query q con respecto a las restricciones definidas en el esquema RDF del \textit{dataset} D en una query q$'$ tal que $q'(D_{data}) = q(D^{\infty})$.\\

La técnica de reformulación de consultas mostrada en las subsecciones anteriores no cumple con lo anterior, a causa de la indistinguibilidad de los \textit{blank nodes}.\\

Para solucionar esto, se define lo siguiente:

\begin{itemize}

\item La \textbf{evaluación no estándar} de una query parcialmente instanciada $q_{\sigma}$ contra un \textit{dataset} D, siendo Var($q_{\sigma}$) el conjunto de variables de q y Val(D) el conjunto de valores de D, se define como:\\
\includegraphics[scale=0.5]{imgs/03.png}\\
Observación: los \textit{blank nodes} quedan intactos.

\item La siguiente \textbf{propiedad} relaciona evaluación y conjunto respuesta estándar y no estándar.\\
Sea D \textit{database} y $q_{\sigma}$ una query parcialmente instanciada contra D, vale:

\includegraphics[scale=0.5]{imgs/04.png}

\item \textbf{Teorema}: (nueva técnica para resolver consultas basada en reformulación) Sea q una query BGP sin \textit{blank nodes} y D un \textit{dataset} cuyo esquema es S, vale:

\begin{center}
\includegraphics[scale=0.55]{imgs/05.png}
\end{center}

\end{itemize}

\subsection{Algoritmo de Reformulación}

El algoritmo de reformulación toma la query original q y las reglas de reformulación detallas anteriormente para generar nuevas queries mediante la aplicación de dichas reglas a los átomos de q. Se procede iterativamente sobre el resultado obtenido hasta que no se obtengan nuevas queries. Finalmente, se devuelve la unión de las nuevas queries generadas y la query original.\\

En cada iteración del algoritmo mostrado a continuación se matchean los átomos de la query con el input de cada regla; de haber coincidencia, se aplica dicha regla.\\

Este algoritmo termina, es correcto y completo. La complejidad de la reformulación de queries es polinomial en el tamaño del esquema y exponencial en el tamaño de la query. La resolución de la consulta contra un \textit{dataset} D está en LogSpace en el tamaño de D y es exponencial en el tamaño de la query.\\

\vspace{100 mm}

\begin{center}
\includegraphics[scale=0.55]{imgs/06.png}
\end{center}

\newpage
\section{Mecanismos de conversión de la Web}
Para obtener una adecuada definición de los datos, la Web Semántica utiliza esencialmente RDF, SPARQL, y OWL, mecanismos que ayudan a convertir la Web en una infraestructura global en la que es posible compartir, y reutilizar datos y documentos entre diferentes tipos de usuarios.

\begin{itemize}
\item \textbf{RDF} proporciona información descriptiva simple sobre los recursos que se encuentran en la Web y que se utiliza, por ejemplo, en catálogos de libros, directorios, colecciones personales de música, fotos, eventos, etc.
\item \textbf{SPARQL} es lenguaje de consulta sobre RDF, que permite hacer búsquedas sobre los recursos de la Web Semántica utilizando distintas fuentes datos.
\item \textbf{OWL} es un mecanismo para desarrollar temas o vocabularios específicos en los que asociar esos recursos. Lo que hace OWL es proporcionar un lenguaje para definir ontologías estructuradas que pueden ser utilizadas a través de diferentes sistemas. Las ontologías, que se encargan de definir los términos utilizados para describir y representar un área de conocimiento, son utilizadas por los usuarios, las bases de datos y las aplicaciones que necesitan compartir información específica, es decir, en un campo determinado como puede ser el de las finanzas, medicina, deporte, etc. Las ontologías incluyen definiciones de conceptos básicos en un campo determinado y la relación entre ellos.
\end{itemize}

\newpage

\section{Referencias}

\begin{itemize}
\item The World Wide Web Consortium (W3C). http://www.w3c.es/Divulgacion/GuiasBreves/WebSemantica
\item http://jplu.developpez.com/tutoriels/web-semantique/introduction/
\end{itemize}

\end{document}